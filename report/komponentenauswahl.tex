\documentclass[a4paper,11pt]{article}

\usepackage[utf8]{inputenc}
\usepackage[T1]{fontenc}
\usepackage[ngerman]{babel}
\usepackage{graphicx}
\usepackage{array}
\usepackage{booktabs}
\usepackage{float}
\usepackage{pifont}
\usepackage{xcolor}
\usepackage{threeparttable}

% Bewertungssymbole
\newcommand{\cmark}{\textcolor{green!70!black}{\ding{51}}}  % Erfüllt
\newcommand{\wmark}{\textcolor{orange}{\ding{115}}}        % Teilweise
\newcommand{\xmark}{\textcolor{red}{\ding{55}}}            % Nicht erfüllt

\usepackage[scaled]{helvet}
\renewcommand{\familydefault}{\sfdefault}

\usepackage[a4paper, top=2cm, bottom=2cm, left=2cm, right=2cm]{geometry}

\usepackage{setspace}
\setstretch{1.5}

\setlength{\parindent}{0pt}
\setlength{\parskip}{6pt}

\usepackage{microtype}
\sloppy

\usepackage{titlesec}
\titleformat{\section}{\normalfont\fontsize{12}{14}\bfseries}{\thesection}{1em}{}
\titleformat{\subsection}{\normalfont\fontsize{12}{14}\bfseries}{\thesubsection}{1em}{}
\titleformat{\subsubsection}{\normalfont\fontsize{11}{13}\bfseries}{\thesubsubsection}{1em}{}

\usepackage[colorlinks=true,linkcolor=black,citecolor=blue,urlcolor=blue]{hyperref}

\usepackage{fancyhdr}
\pagestyle{fancy}
\fancyhf{}
\renewcommand{\headrulewidth}{0pt}
\fancyfoot[C]{\thepage}

\usepackage[german=quotes]{csquotes}

\usepackage{caption}
\captionsetup[table]{font=small,skip=10pt,labelfont=bf}

\usepackage[backend=biber, style=apa]{biblatex}
\addbibresource{references.bib}

\begin{document}

\begin{titlepage}
    \thispagestyle{empty}
    \centering
    \vspace*{5cm}
    {\Huge\bfseries Komponentenauswahl \par}
    \vspace{1cm}
    {\Large KI-basierte Sprachübersetzer im Unternehmenskontext \par}
    \vspace{0.5cm}
    {\large Projekt: Generative KI im Unternehmenskontext (DLBFMPGKIU01) \par}
    \vspace{1cm}
    {\large Sven Behrens \par}
    {\large Matrikelnummer: 42303511 \par}
    \vspace{0.5cm}
    {\large \today \par}
\end{titlepage}

\pagenumbering{arabic}
\setcounter{page}{1}

\section{Einleitung}

Für die Umsetzung des Übersetzungs-Demonstrators ist die Auswahl einer geeigneten Übersetzungskomponente erforderlich. 
Diese Dokumentation beschreibt das systematische Auswahlverfahren basierend auf den definierten Anforderungen.

\section{Untersuchte Optionen}

Folgende Übersetzungslösungen wurden analysiert:

\begin{table}[H]
\centering
\caption{Übersicht der untersuchten Lösungen}
\begin{tabular}{p{4cm}p{3cm}p{6.5cm}}
\toprule
\textbf{Lösung} & \textbf{Typ} & \textbf{Kurzbeschreibung} \\
\midrule
DeepL API (Pro) & Cloud-API & Spezialisierte Übersetzungs-API, Fokus auf EU-Sprachen \parencite{deepl2024llm} \\
Google Cloud Translation & Cloud-API & Breites Sprachangebot (133 Sprachen) \parencite{languageio2025} \\
Microsoft Azure Translator & Cloud-API & Enterprise-Lösung mit Compliance-Fokus \parencite{microsoft2024notrace} \\
Amazon Translate & Cloud-API & AWS-integrierte Übersetzungslösung \parencite{aws2024translate} \\
OpenAI GPT-4 & LLM-API & Generatives Sprachmodell mit Übersetzungsfähigkeit \parencite{getblend2025llm} \\
Anthropic Claude & LLM-API & Generatives Sprachmodell, großer Kontext \parencite{getblend2025llm} \\
Meta NLLB-200 & Open-Source & Vielsprachiges Übersetzungsmodell \parencite{wmt2024findings} \\
Helsinki OPUS-MT & Open-Source & Sprachpaarspezifische Modelle \parencite{wmt2024findings} \\
\bottomrule
\end{tabular}
\end{table}

\section{Bewertung anhand der Anforderungen}

Die Lösungen werden anhand der in der Anforderungsliste definierten Kriterien bewertet.
Die Bewertung erfolgt mit: \cmark{} = erfüllt, \wmark{} = teilweise erfüllt, \xmark{} = nicht erfüllt.

\subsection{Funktionale Anforderungen}

\begin{table}[H]
\centering
\caption{Bewertung funktionaler Anforderungen}
\footnotesize
\begin{tabular}{p{2.5cm}cccccccc}
\toprule
\textbf{Anforderung} & \textbf{DeepL} & \textbf{Azure} & \textbf{Google} & \textbf{Amazon} & \textbf{GPT-4} & \textbf{Claude} & \textbf{NLLB} & \textbf{OPUS} \\
\midrule
REST-API verfügbar & \cmark & \cmark & \cmark & \cmark & \cmark & \cmark & \wmark & \wmark \\
Sprachen DE/EN/FR/ES & \cmark & \cmark & \cmark & \cmark & \cmark & \cmark & \cmark & \cmark \\
Batch-Übersetzung & \cmark & \cmark & \cmark & \cmark & \wmark & \wmark & \wmark & \wmark \\
Spracherkennung & \cmark & \cmark & \cmark & \cmark & \wmark & \wmark & \wmark & \wmark \\
\bottomrule
\end{tabular}
\end{table}

DeepL, Azure, Google und Amazon bieten vollständige REST-APIs mit automatischer Spracherkennung und Batch-Verarbeitung \parencite{deepl2024api, microsoft2024notrace, aws2024translate}.
GPT-4 und Claude erfordern Prompt-basierte Integration; Claude bietet keine dedizierte Batch-API, ermöglicht aber parallele Requests \parencite{anthropic2024api}.
Die Open-Source-Modelle NLLB und OPUS-MT benötigen eigene API-Wrapper und externe Spracherkennungslösungen \parencite{picovoice2025opensource}.





\subsection{Nicht-funktionale Anforderungen}


\begin{table}[H]
\centering
\caption{Bewertung nicht-funktionaler Anforderungen}
\footnotesize
\begin{tabular}{p{2.5cm}cccccccc}
\toprule
\textbf{Anforderung} & \textbf{DeepL} & \textbf{Azure} & \textbf{Google} & \textbf{Amazon} & \textbf{GPT-4} & \textbf{Claude} & \textbf{NLLB} & \textbf{OPUS} \\
\midrule
Übersetzungsqualität & \cmark & \wmark & \cmark & \cmark & \cmark & \cmark & \wmark & \wmark \\
Antwortzeit < 3s & \cmark & \cmark & \cmark & \cmark & \wmark & \wmark & \wmark & \cmark \\
Kosten < 100€/Monat & \wmark & \cmark & \wmark & \cmark & \wmark & \wmark & \cmark & \cmark \\
100 parallele Anfragen & \wmark & \cmark & \cmark & \cmark & \xmark & \wmark & \wmark & \wmark \\
API-Key-Auth & \cmark & \cmark & \cmark & \cmark & \cmark & \cmark & \xmark & \xmark \\
OpenAPI/Swagger-Doku & \cmark & \wmark & \wmark & \wmark & \wmark & \wmark & \xmark & \xmark \\
HTTPS-Verschlüsselung & \cmark & \cmark & \cmark & \cmark & \cmark & \cmark & \wmark & \wmark \\
\bottomrule
\end{tabular}
\end{table}


DeepL und Claude liefern die höchste Übersetzungsqualität; Claude wurde in Tests als „Translation Champion" bezeichnet \parencite{taia2025comparison, getblend2025llm}.
Azure und Amazon bieten das beste Preis-Leistungs-Verhältnis mit ca. 10--15\$/Mio Zeichen und Free-Tier-Optionen \parencite{azure2024limits, aws2024translate}.
GPT-4 und Claude zeigen hohe Qualität, aber längere Antwortzeiten und bei Claude strikte Rate-Limits \parencite{openai2024terms, anthropic2024ratelimits}.
Open-Source-Lösungen erfordern Selbsthosting mit eigener TLS-Konfiguration.

\clearpage
\printbibliography

\end{document}
