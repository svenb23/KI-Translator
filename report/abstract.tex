\documentclass[a4paper,11pt]{article}

\usepackage[utf8]{inputenc}
\usepackage[T1]{fontenc}
\usepackage[ngerman]{babel}
\usepackage{graphicx}

\usepackage[scaled]{helvet}
\renewcommand{\familydefault}{\sfdefault}

\usepackage[a4paper, top=2cm, bottom=2cm, left=2cm, right=2cm]{geometry}

\usepackage{setspace}
\setstretch{1.5}

\setlength{\parindent}{0pt}
\setlength{\parskip}{6pt}

\usepackage{microtype}
\sloppy

\usepackage[german=quotes]{csquotes}

\usepackage[colorlinks=true,linkcolor=black,citecolor=blue,urlcolor=blue]{hyperref}

\usepackage{fancyhdr}
\pagestyle{fancy}
\fancyhf{}
\renewcommand{\headrulewidth}{0pt}
\fancyfoot[C]{\thepage}

\begin{document}

\hypersetup{pageanchor=false}
\begin{titlepage}
    \thispagestyle{empty}
    \centering
    \vspace*{5cm}
    {\Huge\bfseries Abstract \par}
    \vspace{1cm}
    {\Large KI-basierte Sprachübersetzer im Unternehmenskontext \par}
    \vspace{0.5cm}
    {\large Projekt: Generative KI im Unternehmenskontext (DLBFMPGKIU01) \par}
    \vspace{1cm}
    {\large Sven Behrens \par}
    {\large Matrikelnummer: 42303511 \par}
    \vspace{0.5cm}
    {\large \today \par}
\end{titlepage}
\hypersetup{pageanchor=true}

\pagenumbering{arabic}
\setcounter{page}{1}

\section*{Abstract}
\subsection*{Hintergrund und Problemstellung}

Kleine und mittlere Unternehmen (KMU) mit internationalem Kundenstamm stehen vor der Herausforderung, 
Support-Anfragen und Kundenkorrespondenz in mehreren Sprachen effizient zu bearbeiten. Professionelle 
Übersetzungsdienste sind kostenintensiv, manuelle Übersetzungen verlangsamen den Workflow erheblich. 
KI-basierte Übersetzungslösungen versprechen hier Abhilfe, doch die Auswahl der geeigneten Technologie 
ist angesichts der Vielzahl verfügbarer Optionen, von kommerziellen Cloud-APIs bis hin zu 
Open-Source-Modellen, nicht trivial.

\subsection*{Zielsetzung}
Ziel dieses Projekts war die Entwicklung eines funktionsfähigen Demonstrators, der zwei unterschiedliche 
Ansätze zur maschinellen Übersetzung präsentiert: eine kommerzielle Cloud-Lösung (DeepL API) und ein selbstgehostetes 
Open-Source-Modell (Helsinki OPUS-MT). Der Demonstrator sollte als REST-API mit Weboberfläche realisiert werden und 
die Sprachen Deutsch, Englisch, Französisch und Spanisch unterstützen. Dabei waren sowohl funktionale Anforderungen wie 
automatische Spracherkennung als auch nicht-funktionale Anforderungen wie DSGVO-Konformität und 
EU-AI-Act-Transparenzpflichten zu berücksichtigen.

\subsection*{Methodik}
Das Projekt wurde in drei Phasen durchgeführt: Konzeption, Erarbeitung und Finalisierung. In der Konzeptionsphase 
erfolgte eine systematische Anforderungsanalyse mit MoSCoW-Priorisierung.

Für die Implementierung wurde ein Python-Backend mit FastAPI entwickelt, das zwei separate 
Übersetzungs-Endpunkte bereitstellt. Die DeepL-Integration nutzt die offizielle Python-Bibliothek mit 
asynchroner Ausführung. Für OPUS-MT werden die Helsinki-NLP-Modelle über Hugging Face Transformers geladen, 
wobei ein Pivot-Mechanismus über Englisch fehlende direkte Sprachpaare kompensiert. Das Frontend besteht 
aus einer HTML/JavaScript-Weboberfläche mit Darstellung beider Übersetzungsergebnisse.

Zur Evaluierung wurden 30 deutsche Testsätze erstellt, die unterschiedliche sprachliche Herausforderungen abdecken: 
Redewendungen und Idiome (13 Sätze), Fachsprache und komplexe Grammatik (7 Sätze) sowie praxisnahe 
Ticketsystem-Anfragen aus dem IT-Support (10 Sätze). Jeder Satz wurde in drei Zielsprachen übersetzt, 
sodass insgesamt 90 Übersetzungen pro Engine entstanden.

\subsection*{Ergebnisse}
Die Evaluation zeigt deutliche Qualitätsunterschiede zwischen den beiden Lösungen:

DeepL übersetzte alle 90 Testsätze erfolgreich und lieferte durchgehend natürlich klingende Ergebnisse. 
Besonders hervorzuheben ist die Fähigkeit, deutsche Redewendungen idiomatisch korrekt zu übertragen. So wurde 
\enquote{Man soll den Tag nicht vor dem Abend loben} treffend mit \enquote{Don't count your chickens before they hatch} übersetzt.
Auch umgangssprachliche Wendungen wie \enquote{Jetzt geht's um die Wurst} wurden sinngemäß wiedergegeben (\enquote{Now it's crunch time}), 
anstatt wörtlich übersetzt zu werden.

OPUS-MT zeigte deutliche Limitierungen. Bei etwa einem Drittel der Sätze gab das Modell den deutschen Originaltext 
unverändert zurück, da die Spracherkennung diese Sätze fälschlicherweise als Englisch erkannte. 
Redewendungen wurden häufig wörtlich übertragen, was zu sinnentstellenden Ergebnissen führte: \enquote{Jetzt geht's um die Wurst} 
wurde zu \enquote{Now it's about the sausage}. Im Ticketsystem-Kontext übersetzte OPUS-MT \enquote{Ticket} fälschlicherweise 
als \enquote{billet} (Fahrkarte) statt als Support-Anfrage.


\end{document}
