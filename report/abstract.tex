\documentclass[a4paper,11pt]{article}

\usepackage[utf8]{inputenc}
\usepackage[T1]{fontenc}
\usepackage[ngerman]{babel}
\usepackage{graphicx}

\usepackage[scaled]{helvet}
\renewcommand{\familydefault}{\sfdefault}

\usepackage[a4paper, top=2cm, bottom=2cm, left=2cm, right=2cm]{geometry}

\usepackage{setspace}
\setstretch{1.5}

\setlength{\parindent}{0pt}
\setlength{\parskip}{6pt}

\usepackage{microtype}
\sloppy

\usepackage[german=quotes]{csquotes}

\usepackage[colorlinks=true,linkcolor=black,citecolor=blue,urlcolor=blue]{hyperref}

\usepackage{fancyhdr}
\pagestyle{fancy}
\fancyhf{}
\renewcommand{\headrulewidth}{0pt}
\fancyfoot[C]{\thepage}

\begin{document}

\hypersetup{pageanchor=false}
\begin{titlepage}
    \thispagestyle{empty}
    \centering
    \vspace*{5cm}
    {\Huge\bfseries Abstract \par}
    \vspace{1cm}
    {\Large KI-basierte Sprachübersetzer im Unternehmenskontext \par}
    \vspace{0.5cm}
    {\large Projekt: Generative KI im Unternehmenskontext (DLBFMPGKIU01) \par}
    \vspace{1cm}
    {\large Sven Behrens \par}
    {\large Matrikelnummer: 42303511 \par}
    \vspace{0.5cm}
    {\large \today \par}
\end{titlepage}
\hypersetup{pageanchor=true}

\pagenumbering{arabic}
\setcounter{page}{1}

\section*{Abstract}
\subsection*{Hintergrund und Problemstellung}

Kleine und mittlere Unternehmen (KMU) mit internationalem Kundenstamm stehen vor der Herausforderung, 
Support-Anfragen und Kundenkorrespondenz in mehreren Sprachen effizient zu bearbeiten. Professionelle 
Übersetzungsdienste sind kostenintensiv, manuelle Übersetzungen verlangsamen den Workflow erheblich. 
KI-basierte Übersetzungslösungen versprechen hier Abhilfe, doch die Auswahl der geeigneten Technologie 
ist angesichts der Vielzahl verfügbarer Optionen -- von kommerziellen Cloud-APIs bis hin zu 
Open-Source-Modellen -- nicht trivial.

\end{document}
