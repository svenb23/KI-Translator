\documentclass[a4paper,11pt]{article}

\usepackage[utf8]{inputenc}
\usepackage[T1]{fontenc}
\usepackage[ngerman]{babel}
\usepackage{graphicx}
\usepackage{array}
\usepackage{booktabs}
\usepackage{amsmath}
\usepackage{float}
\usepackage{longtable}
\usepackage{pifont}
\usepackage{xcolor}
\usepackage{threeparttable}
\usepackage{ragged2e}

\newcolumntype{L}[1]{>{\RaggedRight\arraybackslash}p{#1}}

% Bewertungssymbole
\newcommand{\cmark}{\textcolor{green!70!black}{\ding{51}}}
\newcommand{\wmark}{\textcolor{orange}{\ding{115}}}
\newcommand{\xmark}{\textcolor{red}{\ding{55}}}

\usepackage[scaled]{helvet}
\renewcommand{\familydefault}{\sfdefault}

\usepackage[a4paper, top=2cm, bottom=2cm, left=2cm, right=2cm]{geometry}

\usepackage{setspace}
\setstretch{1.5}

\setlength{\parindent}{0pt}
\setlength{\parskip}{6pt}

\usepackage{microtype}
\sloppy
\hyphenpenalty=1000
\tolerance=3000

\renewcommand{\footnotesize}{\fontsize{10}{12}\selectfont}

\setcounter{secnumdepth}{3}
\setcounter{tocdepth}{3}

\usepackage{titlesec}
\titleformat{\section}{\normalfont\fontsize{14}{16}\bfseries}{\thesection}{1em}{}
\titleformat{\subsection}{\normalfont\fontsize{12}{14}\bfseries}{\thesubsection}{1em}{}
\titleformat{\subsubsection}{\normalfont\fontsize{11}{13}\bfseries}{\thesubsubsection}{1em}{}

\usepackage[
  colorlinks=true,
  linkcolor=black,
  citecolor=blue,
  filecolor=black,
  urlcolor=blue
]{hyperref}
\usepackage[capitalise,nameinlink]{cleveref}

\usepackage{fancyhdr}
\pagestyle{fancy}
\fancyhf{}
\renewcommand{\headrulewidth}{0pt}
\fancyfoot[C]{\thepage}

\usepackage[backend=biber, style=apa]{biblatex}
\addbibresource{references.bib}

\usepackage{titling}
\usepackage{acronym}
\usepackage[german=quotes]{csquotes}

\usepackage{caption}
\captionsetup[table]{font=small,skip=10pt,labelfont=bf}

\usepackage{listings}

\definecolor{codegreen}{rgb}{0,0.6,0}
\definecolor{codegray}{rgb}{0.5,0.5,0.5}
\definecolor{codepurple}{rgb}{0.58,0,0.82}
\definecolor{backcolour}{rgb}{0.95,0.95,0.92}

\lstdefinestyle{mystyle}{
    backgroundcolor=\color{backcolour},
    commentstyle=\color{codegreen},
    keywordstyle=\color{magenta},
    numberstyle=\tiny\color{codegray},
    stringstyle=\color{codepurple},
    basicstyle=\ttfamily\footnotesize,
    breakatwhitespace=false,
    breaklines=true,
    captionpos=b,
    keepspaces=true,
    numbers=left,
    numbersep=5pt,
    showspaces=false,
    showstringspaces=false,
    showtabs=false,
    tabsize=2
}
\lstset{style=mystyle}

\begin{document}

% ============================================================================
% TITELSEITE
% ============================================================================
\begin{titlepage}
    \thispagestyle{empty}
    \centering
    \vspace*{4cm}
    {\Huge\bfseries Projekt: Generative KI im Unternehmenskontext \par}
    \vspace{0.5cm}
    {\Large (DLBFMPGKIU01) \par}
    \vspace{1.5cm}
    {\LARGE KI-basierte Sprachübersetzer im Unternehmenskontext \par}
    \vspace{1cm}
    {\large Gesamtdokumentation aller Projektphasen \par}
    \vspace{2cm}
    {\large Studiengang: Angewandte Künstliche Intelligenz \par}
    \vspace{0.5cm}
    {\large Sven Behrens \par}
    \vspace{0.5cm}
    {\large Matrikelnummer: 42303511 \par}
    \vspace{0.5cm}
    {\large Tutor: Prof. Dr. Oliver Dorn \par}
    \vspace{1cm}
    {\large \today \par}
\end{titlepage}

% ============================================================================
% INHALTSVERZEICHNIS
% ============================================================================
\pagenumbering{Roman}
\setcounter{page}{1}

\tableofcontents
\newpage

\listoftables
\newpage

\listoffigures
\newpage

% ============================================================================
% ABKÜRZUNGSVERZEICHNIS
% ============================================================================
\section*{Abkürzungsverzeichnis}
\addcontentsline{toc}{section}{Abkürzungsverzeichnis}
\begin{acronym}[DSGVO]
    \acro{API}{Application Programming Interface}
    \acro{DSGVO}{Datenschutz-Grundverordnung}
    \acro{KMU}{Kleine und mittlere Unternehmen}
    \acro{LLM}{Large Language Model}
    \acro{REST}{Representational State Transfer}
    \acro{SLA}{Service-Level Agreement}
\end{acronym}
\newpage

\pagenumbering{arabic}
\setcounter{page}{1}

% ============================================================================
% TEIL I: KONZEPTIONSPHASE
% ============================================================================
\part{Konzeptionsphase}

\section{Konzeptvorstellung}
Kleine und mittlere Unternehmen (KMU) mit internationalem Kundenstamm stehen vor der Herausforderung, Support-Anfragen
in mehreren Sprachen effizient zu bearbeiten. Professionelle Übersetzungsdienste sind kostenintensiv, manuelle Übersetzungen
verlangsamen den Workflow. KI-basierte Übersetzungs-APIs bieten eine Lösung, die sich nahtlos in bestehende Ticket-Systeme integrieren lässt.

Das vorliegende Projekt wird im Rahmen des Moduls \enquote{Projekt: Generative KI im Unternehmenskontext} an der IU
Internationalen Hochschule durchgeführt. Das fiktive Start-up \enquote{TranslateAI} spezialisiert sich auf KI-basierte Übersetzungs-APIs für KMU.

\subsection{Zielsetzung}
Ziel ist die Entwicklung eines kostengünstigen Demonstrators einer Übersetzungs-\ac{API}. Diese ermöglicht die Integration in
Weboberflächen wie z.B. ein Ticket-System, das per Knopfdruck den Text an die \ac{API} sendet und die Übersetzung zurückerhält.
Unterstützt werden Deutsch, Englisch, Französisch und Spanisch.

\subsection{Methodisches Vorgehen}
Das Projekt folgt einem Drei-Phasen-Zyklus: In der Konzeptionsphase erfolgen Anforderungsanalyse und Technologierecherche.
Die Erarbeitungsphase umfasst die Auswahl geeigneter LLM-Komponenten sowie die Implementierung der REST-API und Weboberfläche.
In der Finalisierungsphase werden Funktions- und Qualitätstests durchgeführt und die Ergebnisse dokumentiert.

Für den Demonstrator wird FastAPI mit einfachem HTML/JavaScript eingesetzt, da dies für einen Prototyp
ausreichend ist und schnelle Entwicklung ermöglicht. Ein späterer Umstieg auf ein umfangreicheres Framework wie Django und Vue.js
wäre bei Bedarf möglich.

\subsection{Zielgruppe}
Der Demonstrator richtet sich an KMU mit 10--250 Mitarbeitern und internationalem Kundenkontakt. Typische Anwendungsfälle
sind die Übersetzung von Support-Tickets und Kundenanfragen. Die \ac{API} ermöglicht die Integration in Ticket-Systeme und Webanwendungen.

\subsection{Tragfähigkeitsprüfung}
Für das Konzept sprechen der wachsende Marktbedarf bei KMU, die technologische Reife aktueller LLMs sowie kosteneffiziente Pay-per-Use-Modelle.
REST-APIs sind Industriestandard und ermöglichen einfache Integration.

Risiken bestehen bei der Übersetzungsqualität (Halluzinationen bei Fachbegriffen), Vendor Lock-in und Datenschutz.

\newpage

% ============================================================================
% ANFORDERUNGSLISTE
% ============================================================================
\section{Priorisierte Anforderungsliste}

\subsection{Ziel-KMU}

\begin{table}[H]
\centering
\caption{Charakteristiken der Ziel-KMU}
\begin{tabular}{p{4cm}p{10cm}}
\toprule
\textbf{Merkmal} & \textbf{Beschreibung} \\
\midrule
Branche & Dienstleistung, E-Commerce, IT-Services mit internationalem Kundenstamm \\
Unternehmensgröße & 10--250 Mitarbeiter (EU-Definition KMU) \\
IT-Kompetenz & Grundlegende IT-Kenntnisse, Erfahrung mit Ticket-Systemen \\
Übersetzungsbedarf & Support-Tickets, Kundenanfragen, E-Mail-Korrespondenz \\
Systemlandschaft & Nutzung gängiger Ticket- oder CRM-Systeme (z.B. Zendesk, Freshdesk) \\
Budget & Begrenzt, Fokus auf kosteneffiziente Lösungen \\
Voraussetzungen & Internetzugang und Browser oder API-fähiges System \\
\bottomrule
\end{tabular}
\end{table}

\subsection{Funktionale Anforderungen}

\begin{table}[H]
\centering
\caption{Funktionale Anforderungen}
\begin{tabular}{p{1.2cm}p{7cm}p{5cm}}
\toprule
\textbf{Priorität} & \textbf{Beschreibung} & \textbf{Begründung} \\
\midrule
Must & REST-API für Übersetzungen zwischen DE, EN, FR, ES & Kernfunktion des Demonstrators \\
Must & API-Endpunkt mit Parametern für Quell-/Zielsprache und Text & Voraussetzung für Ticket-System-Integration \\
Must & Weboberfläche mit Textfeld, Sprachauswahl und Übersetzungs-Button & Direkte Nutzung für die Vorstellung \\
Must & Standardisierte Status- und Fehlermeldungen (HTTP-Codes, JSON) & Notwendig für stabile Systemintegration \\
Should & Erhalt von Textformatierungen (Absätze, Listen) & Wichtig für strukturierte Tickets \\
Should & Automatische Erkennung der Quellsprache, falls nicht angegeben & Reduziert Bedienaufwand und Fehlerquellen \\
Should & Übersetzung mehrerer Texte in einer Anfrage (Batch) & Effizienzsteigerung bei Ticket-Verarbeitung \\
Should & Protokollierung von Anfragen ohne Speicherung des Inhalts & Ermöglicht Debugging und Nachvollziehbarkeit \\
Could & Callback-URLs für asynchrone Verarbeitung & Für längere Texte hilfreich \\
Could & Dashboard für API-Nutzung und Tokenverbrauch & Kostenkontrolle \\
Won't & Weitere Sprachen (z.B. Chinesisch, Japanisch) & Außerhalb des Projektumfangs \\
Won't & Simultanübersetzung für Videokonferenzen & Technisch zu komplex \\
Won't & Speech-to-Text und Text-to-Speech & Anderer Anwendungsfall \\
Won't & PDF/Word-Verarbeitung mit Layout-Erhalt & Erhöhte Komplexität \\
\bottomrule
\end{tabular}
\end{table}

\subsection{Nicht-funktionale Anforderungen}

\begin{table}[H]
\centering
\caption{Nicht-funktionale Anforderungen}
\begin{tabular}{p{1.2cm}p{7cm}p{5cm}}
\toprule
\textbf{Priorität} & \textbf{Beschreibung} & \textbf{Begründung} \\
\midrule
Must & Übersetzung muss verständlich und sinngemäß korrekt sein & Qualität entscheidet über Nutzbarkeit \\
Should & Antwortzeit der API unter 3 Sekunden & Akzeptanz im Arbeitsalltag \\
Should & Unterstützung von mind. 100 parallelen Anfragen & Relevanz für KMU-Einsatz \\
Should & OpenAPI/Swagger-Dokumentation & Einfache Integration \\
Should & API-Key-basierte Authentifizierung & Zugriffskontrolle \\
Should & HTTPS-Verschlüsselung für alle Übertragungen & Datenschutz \\
Should & Betriebskosten unter 100€/Monat bei typischem KMU-Volumen & Budget der Zielgruppe \\
Could & Konfigurierbare Rate-Limits pro API-Key & Missbrauchsschutz \\
Could & Containerisierbar (z.B. Docker) & Flexible Bereitstellung \\
\bottomrule
\end{tabular}
\end{table}

\subsection{Rechtliche und ethische Anforderungen}

\begin{table}[H]
\centering
\caption{Rechtliche und ethische Anforderungen}
\begin{tabular}{p{1.2cm}p{7cm}p{5cm}}
\toprule
\textbf{Priorität} & \textbf{Beschreibung} & \textbf{Begründung} \\
\midrule
Must & Keine dauerhafte Speicherung von Übersetzungstexten & \ac{DSGVO}-Konformität \parencite{dsgvo2016} \\
Must & Verwendete LLM-API muss kommerziell nutzbar sein & Unternehmenskontext \\
Must & Einstufung des Systems als Niedrigrisiko-KI gemäß EU AI Act \parencite{euaiact2024} & Regulatorische Einordnung \\
Must & Eingabetexte verbleiben im Eigentum des Nutzers & Rechtssicherheit \\
Should & Transparente Dokumentation der Datenverarbeitung & Vertrauensbildung \\
Should & Kennzeichnung als KI-generierte Übersetzung & EU AI-Act Transparenzpflicht \\
Should & Keine Nutzung der Texte für Modelltraining & Datenschutz und IP-Schutz \\
Should & KI darf keine zusätzlichen Inhalte halluzinieren & Risiko bei LLM-Übersetzungen \\
\bottomrule
\end{tabular}
\end{table}

\newpage

% ============================================================================
% TEIL II: ERARBEITUNGSPHASE
% ============================================================================
\part{Erarbeitungsphase}

\section{Komponentenauswahl}

\subsection{Einleitung}
Für die Umsetzung des Übersetzungs-Demonstrators ist die Auswahl einer geeigneten Übersetzungskomponente erforderlich.
Diese Dokumentation beschreibt das systematische Auswahlverfahren basierend auf den definierten Anforderungen.

\subsection{Untersuchte Optionen}

Folgende Übersetzungslösungen wurden analysiert:

\begin{table}[H]
\centering
\caption{Übersicht der untersuchten Lösungen}
\begin{tabular}{p{4cm}p{3cm}p{6.5cm}}
\toprule
\textbf{Lösung} & \textbf{Typ} & \textbf{Kurzbeschreibung} \\
\midrule
DeepL API (Pro) & Cloud-API & Spezialisierte Übersetzungs-API, Fokus auf EU-Sprachen \parencite{deepl2024llm} \\
Google Cloud Translation & Cloud-API & Breites Sprachangebot (133 Sprachen) \parencite{languageio2025} \\
Microsoft Azure Translator & Cloud-API & Enterprise-Lösung mit Compliance-Fokus \parencite{microsoft2024notrace} \\
Amazon Translate & Cloud-API & AWS-integrierte Übersetzungslösung \parencite{aws2024translate} \\
OpenAI GPT-4 & LLM-API & Generatives Sprachmodell mit Übersetzungsfähigkeit \parencite{getblend2025llm} \\
Anthropic Claude & LLM-API & Generatives Sprachmodell, großer Kontext \parencite{getblend2025llm} \\
Meta NLLB-200 & Open-Source & Vielsprachiges Übersetzungsmodell \parencite{wmt2024findings} \\
Helsinki OPUS-MT & Open-Source & Sprachpaarspezifische Modelle \parencite{wmt2024findings} \\
\bottomrule
\end{tabular}
\end{table}

\subsection{Bewertung anhand der Anforderungen}

Die Lösungen werden anhand der in der Anforderungsliste definierten Kriterien bewertet.
Die Bewertung erfolgt mit: \cmark{} = erfüllt, \wmark{} = teilweise erfüllt, \xmark{} = nicht erfüllt.

\subsubsection{Funktionale Anforderungen}

\begin{table}[H]
\centering
\caption{Bewertung funktionaler Anforderungen}
\footnotesize
\begin{tabular}{p{2.5cm}cccccccc}
\toprule
\textbf{Anforderung} & \textbf{DeepL} & \textbf{Azure} & \textbf{Google} & \textbf{Amazon} & \textbf{GPT-4} & \textbf{Claude} & \textbf{NLLB} & \textbf{OPUS} \\
\midrule
REST-API verfügbar & \cmark & \cmark & \cmark & \cmark & \cmark & \cmark & \wmark & \wmark \\
Sprachen DE/EN/FR/ES & \cmark & \cmark & \cmark & \cmark & \cmark & \cmark & \cmark & \cmark \\
Batch-Übersetzung & \cmark & \cmark & \cmark & \cmark & \wmark & \wmark & \wmark & \wmark \\
Spracherkennung & \cmark & \cmark & \cmark & \cmark & \wmark & \wmark & \wmark & \wmark \\
\bottomrule
\end{tabular}
\end{table}

DeepL, Azure, Google und Amazon bieten vollständige REST-APIs mit automatischer Spracherkennung und Batch-Verarbeitung \parencite{deepl2024api, microsoft2024notrace, aws2024translate}.
GPT-4 und Claude erfordern Prompt-basierte Integration; Claude bietet keine dedizierte Batch-API, ermöglicht aber parallele Requests \parencite{anthropic2024api}.
Die Open-Source-Modelle NLLB und OPUS-MT benötigen eigene API-Wrapper und externe Spracherkennungslösungen \parencite{picovoice2025opensource}.

\subsubsection{Nicht-funktionale Anforderungen}

\begin{table}[H]
\centering
\caption{Bewertung nicht-funktionaler Anforderungen}
\footnotesize
\begin{tabular}{p{2.5cm}cccccccc}
\toprule
\textbf{Anforderung} & \textbf{DeepL} & \textbf{Azure} & \textbf{Google} & \textbf{Amazon} & \textbf{GPT-4} & \textbf{Claude} & \textbf{NLLB} & \textbf{OPUS} \\
\midrule
Übersetzungsqualität & \cmark & \wmark & \cmark & \cmark & \cmark & \cmark & \wmark & \wmark \\
Antwortzeit < 3s & \cmark & \cmark & \cmark & \cmark & \wmark & \wmark & \wmark & \cmark \\
Kosten < 100€/Monat & \wmark & \cmark & \wmark & \cmark & \wmark & \wmark & \cmark & \cmark \\
100 parallele Anfragen & \wmark & \cmark & \cmark & \cmark & \xmark & \wmark & \wmark & \wmark \\
API-Key-Auth & \cmark & \cmark & \cmark & \cmark & \cmark & \cmark & \xmark & \xmark \\
OpenAPI/Swagger-Doku & \cmark & \wmark & \wmark & \wmark & \wmark & \wmark & \xmark & \xmark \\
HTTPS-Verschlüsselung & \cmark & \cmark & \cmark & \cmark & \cmark & \cmark & \wmark & \wmark \\
\bottomrule
\end{tabular}
\end{table}

DeepL und Claude liefern die höchste Übersetzungsqualität. Claude wurde in Tests als „Translation Champion" bezeichnet \parencite{taia2025comparison, getblend2025llm}.
Azure und Amazon bieten das beste Preis-Leistungs-Verhältnis mit ca. 10--15\$/Mio Zeichen und Free-Tier-Optionen \parencite{azure2024limits, aws2024translate}.

\subsubsection{Rechtliche und ethische Anforderungen}

\begin{table}[H]
\centering
\caption{Bewertung rechtlicher und ethischer Anforderungen}
\footnotesize
\begin{tabular}{p{2.5cm}cccccccc}
\toprule
\textbf{Anforderung} & \textbf{DeepL} & \textbf{Azure} & \textbf{Google} & \textbf{Amazon} & \textbf{GPT-4} & \textbf{Claude} & \textbf{NLLB} & \textbf{OPUS} \\
\midrule
DSGVO-konform & \cmark & \cmark & \cmark & \wmark & \wmark & \cmark & \cmark & \cmark \\
Kein Modelltraining & \cmark & \cmark & \cmark & \wmark & \cmark & \cmark & \cmark & \cmark \\
Daten-Eigentum Nutzer & \cmark & \cmark & \cmark & \cmark & \cmark & \cmark & \cmark & \cmark \\
Kein Halluzinieren & \cmark & \cmark & \cmark & \cmark & \wmark & \wmark & \cmark & \cmark \\
Kommerzielle Nutzung & \cmark & \cmark & \cmark & \cmark & \cmark & \cmark & \xmark & \cmark \\
Transparente Datenverarb. & \cmark & \cmark & \cmark & \cmark & \wmark & \cmark & \cmark & \cmark \\
\bottomrule
\end{tabular}
\end{table}

DeepL, Azure, Google und Claude sind DSGVO-konform und nutzen keine Kundendaten zum Modelltraining \parencite{deepl2024privacy, microsoft2024notrace, google2024datausage, anthropic2024privacy}.
Meta NLLB ist unter CC-BY-NC 4.0 lizenziert und daher nicht kommerziell nutzbar \parencite{picovoice2025opensource}.
OPUS-MT-Modelle sind unter CC-BY 4.0 frei verfügbar.

\subsection{Zusammenfassung und Empfehlung}

\begin{table}[H]
\centering
\caption{Gesamtbewertung der Lösungen}
\small
\begin{tabular}{p{4cm}ccp{6cm}}
\toprule
\textbf{Lösung} & \textbf{Eignung} & \textbf{Rang} & \textbf{Begründung} \\
\midrule
DeepL API (Pro) & \cmark & 1 & Höchste Qualität für DE/EN/FR/ES, DSGVO-konform \\
Microsoft Azure Translator & \cmark & 2 & Bestes Preis-Leistungs-Verhältnis, skalierbar \\
Amazon Translate & \cmark & 3 & Günstig, schnell, aber Opt-Out für DSGVO nötig \\
Google Cloud Translation & \wmark & 4 & Solide Qualität, aber teurer als Azure/Amazon \\
Anthropic Claude & \wmark & 5 & Sehr hohe Qualität, aber langsamer und Rate-Limits \\
Helsinki OPUS-MT & \wmark & 6 & Open-Source-Alternative, mehr Aufwand \\
OpenAI GPT-4 & \xmark & -- & Zu langsam, hohe Kosten, strikte API-Limits \\
Meta NLLB-200 & \xmark & -- & Nicht kommerziell nutzbar (CC-BY-NC) \\
\bottomrule
\end{tabular}
\end{table}

Für den Demonstrator werden DeepL und OPUS-MT implementiert.

\textbf{DeepL API} als Qualitätsführer: DeepL bietet die höchste Übersetzungsqualität für europäische Sprachen,
vollständige DSGVO-Konformität mit Serverstandort Deutschland und eine gut dokumentierte REST-API.

\textbf{Helsinki OPUS-MT} als Open-Source-Alternative: OPUS-MT ermöglicht einen direkten Vergleich mit einer
selbstgehosteten Lösung ohne externe API-Abhängigkeit \parencite{reddit2022nllb}. Die Modelle sind unter CC-BY 4.0
frei kommerziell nutzbar und können lokal über Hugging Face Transformers betrieben werden.

\newpage

\section{Erläuterung zur Umsetzung}

\subsection{Architektur}
Das Backend besteht aus einer FastAPI-basierten REST-API mit zwei separaten Übersetzungs-Endpunkten
(\texttt{/translate/deepl} und \texttt{/translate/opus}) sowie einem Health-Check-Endpunkt. Die DeepL-Integration
nutzt die offizielle Python-Bibliothek \texttt{deepl} mit asynchroner Ausführung über \texttt{asyncio}.
Die automatische Spracherkennung wird direkt von der DeepL-API bereitgestellt.

Für OPUS-MT werden die Helsinki-NLP-Modelle über Hugging Face Transformers geladen. Da nicht alle Sprachpaare
direkt verfügbar sind, wurde ein Pivot-Mechanismus implementiert: Übersetzungen wie DE$\rightarrow$FR erfolgen über den
Zwischenschritt DE$\rightarrow$EN$\rightarrow$FR. Ein LRU-Cache speichert bis zu 12 geladene OPUS-MT-Modelle, um wiederholte Ladezeiten zu vermeiden.

Das Frontend besteht aus einer HTML/JavaScript-Weboberfläche mit zwei nebeneinander angeordneten Übersetzungsfeldern, eines für jedes Modell (siehe Abbildung~\ref{fig:frontend}).
Die Sprachauswahl erfolgt über Buttons für die vier unterstützten Sprachen (DE, EN, FR, ES).

\subsection{EU AI Act Konformität}

Die API-Responses enthalten das Feld \texttt{ai\_generated: true} zur Kennzeichnung KI-generierter Inhalte gemäß EU AI Act Transparenzpflicht.
Das System wird als Niedrigrisiko-KI eingestuft, da es ausschließlich für Übersetzungsaufgaben ohne autonome Entscheidungsfindung eingesetzt wird.

\subsection{Abgrenzung}

Der Demonstrator erfüllt alle Must-Anforderungen der Anforderungsliste. Einige Should-Anforderungen wie API-Key-basierte Authentifizierung,
HTTPS-Verschlüsselung, Batch-Übersetzung und Request-Logging wurden bewusst nicht implementiert, da diese für einen Demonstrator den Rahmen
sprengen würden. Diese Funktionen wären für einen produktiven Einsatz erforderlich, sind aber für die Demonstration der Kernfunktionalität nicht notwendig.

\vspace{0.5cm}
\textbf{Repository:} \url{https://github.com/svenb23/KI-Translator}

\newpage

% ============================================================================
% TEIL III: FINALISIERUNGSPHASE
% ============================================================================
\part{Finalisierungsphase}

\section{Testaufbau}

Zur Evaluierung des Demonstrators wurden in Zusammenarbeit mit ChatGPT \cite{chatgpt2024} 30 deutsche Testsätze
erstellt und ausgewählt, die unterschiedliche sprachliche Herausforderungen abdecken: Redewendungen und Idiome
(13 Sätze), Fachsprache und komplexe Grammatik (7 Sätze) sowie praxisnahe Ticketsystem-Anfragen aus dem IT-Support (10 Sätze).
Jeder Satz wurde in drei Zielsprachen übersetzt (Englisch, Französisch, Spanisch), sodass insgesamt 90 Übersetzungen pro
Engine entstanden. Die ChatGPT-Übersetzungen dienen dabei als Referenz für eine menschenähnliche Übersetzungsqualität.

\section{Diskussion der Ergebnisse}

\subsection{Stärken von DeepL}

DeepL übersetzte alle 90 Testsätze erfolgreich und lieferte durchgehend natürlich klingende Ergebnisse.
Besonders hervorzuheben ist die Fähigkeit, deutsche Redewendungen idiomatisch korrekt zu übertragen. So
wurde \enquote{Man soll den Tag nicht vor dem Abend loben} (Satz~3) treffend mit \enquote{Don't count your chickens before they hatch}
übersetzt, ein englisches Sprichwort mit derselben Bedeutung. Auch umgangssprachliche Wendungen
wie \enquote{Jetzt geht's um die Wurst} (Satz~2) wurden sinngemäß wiedergegeben (\enquote{Now it's crunch time}), anstatt wörtlich übersetzt zu werden.

\subsection{Schwächen von OPUS-MT}

OPUS-MT zeigte deutliche Limitierungen, die auf zwei Hauptursachen zurückzuführen sind:
eine unzureichende Spracherkennung und das Fehlen von idiomatischem Wissen.

Bei etwa einem Drittel der Sätze gab OPUS-MT den deutschen Originaltext unverändert zurück.
Dies betraf insbesondere Sätze ohne eindeutige deutsche Schlüsselwörter wie \enquote{Ich verstehe nur Bahnhof} (Satz~1)
oder \enquote{Ich glaub, ich spinne!} (Satz~12). Die im Demonstrator implementierte
Keyword-basierte Spracherkennung erkannte diese Sätze fälschlicherweise als Englisch, wodurch keine Übersetzung erfolgte.

Redewendungen wurden häufig wörtlich übertragen, was zu sinnentstellenden Ergebnissen führte. \enquote{Jetzt geht's um die Wurst} (Satz~2)
wurde zu \enquote{Now it's about the sausage}, \enquote{Sie hat nicht alle Tassen im Schrank} (Satz~20)
zu \enquote{She doesn't have all the cups in the closet}.

Im Ticketsystem-Kontext übersetzte OPUS-MT \enquote{Ticket} (Satz~21, 25) fälschlicherweise
als \enquote{billet} (Fahrkarte) statt \enquote{ticket} (Support-Anfrage).

\subsection{Empfehlung für den KMU-Einsatz}
Für kleine und mittlere Unternehmen lässt sich aus den Testergebnissen eine klare Empfehlung ableiten:

DeepL eignet sich für alle produktiven Anwendungsfälle, insbesondere für Kundenkommunikation. Die Kosten sind für die gebotene Qualität angemessen.

OPUS-MT kann als Alternative für einfache, eindeutige Texte dienen, bei denen keine
Redewendungen oder umgangssprachlichen Formulierungen vorkommen. Für den produktiven Einsatz in der Kundenkommunikation ist das Modell jedoch nicht geeignet.
Falls der Kunde eine lokal gehostete Anwendung wünscht, sollten die Mitarbeiter speziell für den Umgang mit Fehlern des Modells geschult werden.

\newpage

% ============================================================================
% ANHANG
% ============================================================================
\appendix
\part*{Anhang}
\addcontentsline{toc}{part}{Anhang}

\section{Frontend-Screenshot}

\begin{figure}[H]
\centering
\includegraphics[width=\textwidth]{Frontend.png}
\caption{Weboberfläche des Demonstrators mit zwei nebeneinander angeordneten Übersetzungsfeldern für DeepL und OPUS-MT}
\label{fig:frontend}
\end{figure}

\newpage

\section{Testergebnisse: Vollständiger Vergleich}

Getestet wurden 30 deutsche Sätze mit Übersetzung in drei Zielsprachen (EN, FR, ES).
Verglichen werden DeepL, OPUS-MT und eine ChatGPT-Referenzübersetzung.

\subsection{Redewendungen und Idiome (Sätze 1--13)}

\small\setstretch{1.0}
\begin{longtable}{L{0.22\textwidth}L{0.24\textwidth}L{0.24\textwidth}L{0.24\textwidth}}
\toprule
\textbf{Deutsch} & \textbf{Referenz} & \textbf{DeepL} & \textbf{OPUS-MT} \\
\midrule
\endfirsthead
\toprule
\textbf{Deutsch} & \textbf{Referenz} & \textbf{DeepL} & \textbf{OPUS-MT} \\
\midrule
\endhead
\bottomrule
\endfoot
\bottomrule
\endlastfoot

\multicolumn{4}{l}{\textbf{[01] Ich verstehe nur Bahnhof.}} \\
EN & It's all Greek to me. & I don't understand a word. & Ich verstehe nur Bahnhof. \\
FR & C'est du chinois pour moi. & Je n'y comprends rien. & Ich Verstehe nur Bahnhof. \\
ES & No entiendo ni jota. & No entiendo nada. & Ich verstehe nur Bahnhof. \\
\midrule

\multicolumn{4}{l}{\textbf{[02] Jetzt geht's um die Wurst.}} \\
EN & Now it's do or die. & Now it's crunch time. & Now it's about the sausage. \\
FR & Maintenant, ça passe ou ça casse. & Maintenant, c'est le moment décisif. & Maintenant, c'est à propos de la saucisse. \\
ES & Ahora va en serio. & Ahora es el momento decisivo. & Ahora se trata de la salchicha. \\
\midrule

\multicolumn{4}{l}{\textbf{[03] Man soll den Tag nicht vor dem Abend loben.}} \\
EN & Don't count your chickens before they hatch. & Don't count your chickens before they hatch. & Don't praise the day before the evening. \\
FR & Il ne faut pas crier victoire trop tôt. & Il ne faut pas vendre la peau de l'ours avant de l'avoir tué. & Ne louez pas la veille du soir. \\
ES & No cantes victoria antes de tiempo. & No hay que alabar el día antes de que termine. & No alaben el día antes de la noche. \\
\end{longtable}
\normalsize\setstretch{1.5}

\textit{Hinweis: Die vollständige Tabelle mit allen 30 Testsätzen befindet sich im separaten Dokument \enquote{Finalisierung und Testergebnisse}.}

\newpage

% ============================================================================
% LITERATURVERZEICHNIS
% ============================================================================
\printbibliography[title=Quellenverzeichnis]

\end{document}
