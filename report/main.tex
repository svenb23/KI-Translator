\documentclass[a4paper,11pt]{article}

\usepackage[utf8]{inputenc}
\usepackage[T1]{fontenc}
\usepackage[ngerman]{babel}
\usepackage{graphicx}
\usepackage{array}
\usepackage{booktabs}
\usepackage{amsmath}
\usepackage{float}

\usepackage[scaled]{helvet}
\renewcommand{\familydefault}{\sfdefault}

\usepackage[a4paper, top=2cm, bottom=2cm, left=2cm, right=2cm]{geometry}

\usepackage{setspace}
\setstretch{1.5}

\setlength{\parindent}{0pt}
\setlength{\parskip}{6pt}

\usepackage{microtype}
\sloppy
\hyphenpenalty=1000
\tolerance=3000

\renewcommand{\footnotesize}{\fontsize{10}{12}\selectfont}

\setcounter{secnumdepth}{3}
\setcounter{tocdepth}{3}

\usepackage{titlesec}
\titleformat{\section}{\normalfont\fontsize{12}{14}\bfseries}{\thesection}{1em}{}
\titleformat{\subsection}{\normalfont\fontsize{12}{14}\bfseries}{\thesubsection}{1em}{}
\titleformat{\subsubsection}{\normalfont\fontsize{12}{14}\bfseries}{\thesubsubsection}{1em}{}

\usepackage[
  colorlinks=true,
  linkcolor=black,
  citecolor=blue,
  filecolor=black,
  urlcolor=blue
]{hyperref}
\usepackage[capitalise,nameinlink]{cleveref}

\usepackage{fancyhdr}
\pagestyle{fancy}
\fancyhf{}
\renewcommand{\headrulewidth}{0pt}
\fancyfoot[C]{\thepage}

\usepackage[backend=biber, style=apa]{biblatex}
\addbibresource{references.bib}

\usepackage{titling}

\usepackage{acronym}
\usepackage[german=quotes]{csquotes}

\usepackage{caption}
\usepackage{threeparttable}
\captionsetup[table]{
    font=small,
    skip=10pt,
    labelfont=bf
}

\usepackage{listings}
\usepackage{xcolor}

\definecolor{codegreen}{rgb}{0,0.6,0}
\definecolor{codegray}{rgb}{0.5,0.5,0.5}
\definecolor{codepurple}{rgb}{0.58,0,0.82}
\definecolor{backcolour}{rgb}{0.95,0.95,0.92}

\lstdefinestyle{mystyle}{
    backgroundcolor=\color{backcolour},
    commentstyle=\color{codegreen},
    keywordstyle=\color{magenta},
    numberstyle=\tiny\color{codegray},
    stringstyle=\color{codepurple},
    basicstyle=\ttfamily\footnotesize,
    breakatwhitespace=false,
    breaklines=true,
    captionpos=b,
    keepspaces=true,
    numbers=left,
    numbersep=5pt,
    showspaces=false,
    showstringspaces=false,
    showtabs=false,
    tabsize=2
}
\lstset{style=mystyle}

\begin{document}

\begin{titlepage}
    \thispagestyle{empty}
    \centering
    \vspace*{5cm}
    {\Huge\bfseries Projekt: Generative KI im Unternehmenskontext (DLBFMPGKIU01) \par}
    \vspace{1cm}
    {\Large KI-basierte Sprachübersetzer im Unternehmenskontext \par}
    \vspace{0.5cm}
    {\large Studiengang: Angewandte Künstliche Intelligenz \par}
    \vspace{0.5cm}
    {\large Sven Behrens \par}
    \vspace{0.5cm}
    {\large Matrikelnummer: 42303511 \par}
    \vspace{0.5cm}
    {\large Tutor:in: Prof. Dr. Oliver Dorn \par}
    \vspace{0.5cm}
    {\large \today \par}
\end{titlepage}

\pagenumbering{Roman}
\setcounter{page}{1}

\tableofcontents
\newpage

\listoffigures
\addcontentsline{toc}{section}{Abbildungsverzeichnis}
\newpage

\listoftables
\addcontentsline{toc}{section}{Tabellenverzeichnis}
\newpage

\section*{Abkürzungsverzeichnis}
\addcontentsline{toc}{section}{Abkürzungsverzeichnis}
\begin{acronym}[KMU]
    \acro{API}{Application Programming Interface}
\end{acronym}
\newpage

\pagenumbering{arabic}
\setcounter{page}{1}

\section{Einleitung}
Kleine und mittlere Unternehmen (KMU) mit internationalem Kundenstamm stehen vor der Herausforderung, Support-Anfragen 
in mehreren Sprachen effizient zu bearbeiten. Professionelle Übersetzungsdienste sind kostenintensiv, manuelle Übersetzungen 
verlangsamen den Workflow. KI-basierte Übersetzungs-APIs bieten eine Lösung, die sich nahtlos in bestehende Ticket-Systeme integrieren lässt.

Das vorliegende Projekt wird im Rahmen des Moduls \enquote{Projekt: Generative KI im Unternehmenskontext} an der IU 
Internationalen Hochschule durchgeführt. Das fiktive Start-up \enquote{TranslateAI} spezialisiert sich auf KI-basierte Übersetzungs-APIs für KMU.

\subsection{Zielsetzung}
Ziel ist die Entwicklung eines kostengünstigen Demonstrators einer Übersetzungs-\ac{API}. Diese ermöglicht die Integration in 
Weboberflächen wie z.B. ein Ticket-System, das per Knopfdruck den Text an die \ac{API} sendet und die Übersetzung zurückerhält. 
Unterstützt werden Deutsch, Englisch, Französisch und Spanisch.

\subsection{Methodisches Vorgehen}
Das Projekt folgt einem Drei-Phasen-Zyklus: In der Konzeptionsphase erfolgen Anforderungsanalyse und Technologierecherche. 
Die Erarbeitungsphase umfasst die Auswahl geeigneter LLM-Komponenten sowie die Implementierung der REST-API und Weboberfläche. 
In der Finalisierungsphase werden Funktions- und Qualitätstests durchgeführt und die Ergebnisse dokumentiert.

\subsection{Zielgruppe}
Der Demonstrator richtet sich an KMU mit 10--250 Mitarbeitern und internationalem Kundenkontakt. Typische Anwendungsfälle 
sind die Übersetzung von Support-Tickets und Kundenanfragen. Die \ac{API} ermöglicht die Integration in Ticket-Systeme und Webanwendungen.

\subsection{Tragfähigkeitsprüfung}
Für das Konzept sprechen der wachsende Marktbedarf bei KMU, die technologische Reife aktueller LLMs sowie kosteneffiziente Pay-per-Use-Modelle. 
REST-APIs sind Industriestandard und ermöglichen einfache Integration.

Risiken bestehen bei der Übersetzungsqualität (Halluzinationen bei Fachbegriffen), Vendor Lock-in und Datenschutz.
\clearpage
\section*{Projektrepository}
\addcontentsline{toc}{section}{Projektrepository}
Der vollständige Quellcode des Demonstrators ist im GitHub-Repository verfügbar:
\url{https://github.com/svenb23/KI-Translator}

\newpage
\printbibliography
\addcontentsline{toc}{section}{Literaturverzeichnis}

\newpage
\section*{Verzeichnis der Anhänge}
\addcontentsline{toc}{section}{Verzeichnis der Anhänge}

\appendix
\section*{Anhang}
\addcontentsline{toc}{section}{Anhang}


\end{document}